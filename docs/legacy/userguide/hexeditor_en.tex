\section{HexEditor}\label{sec:hexeditor}

How a file can be opened in HexEditor within \codeblocks.

\begin{enumerate}
\item \menu{File, Open with HexEditor}
\item Project Navigator context menu (\menu{Open with, Hex editor}
\item Select the Tab Files in the Management Panel. By selecting a file in the FileManager and executing the context menu \menu{Open With Hex editor} this file is opened in HexEditor.
\end{enumerate}

Division of windows:

left is HexEditor view and right is the display as strings

Upper line:
Current position (value in decimal/hex) and percentage (ratio of current cursor position to whole file).

Buttons:

Search functions

Goto Button: Jump to an absolute position. Format in decimal or hex. Relative jump forward or backward by specifying a sign.

Search: Search for hex patterns in the HexEditor view or for strings in the file preview.

Configuration of the number of columns:
Exactly, Multiple of, Power of

Display mode:
Hex, Binary

Bytes:
Select how many bytes should be displayed per column.

Choice of Endianess:
BE: Big Endian
LE: Little Endian

Value Preview:
Adds an additional view in HexEditor. For a selected value in HexEditor, the value is also displayed as Word, Dword, Float, Double.

Expression Input:
Allows you to perform an arithmetic operation on a value in HexEditor. Result of the operation is displayed at the right margin.

Calc:
Expression Tester

Editing a file in the HexEditor:

Includes Undo and Redo History.

For example, move the cursor to the string view:
Insert spaces with the Insert key.
Delete characters by pressing the Del key.

By entering a text, the existing content is overwritten as a string.

By entering numbers in the HexEditor view the values are overwritten and the preview is updated.

