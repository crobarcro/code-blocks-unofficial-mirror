\section{Code statistics}

\screenshot{code_stats}{Configuration de Code Statistics}

Basée sur les caractéristiques d'un masque de configuration, cette simple extension détecte les pourcentages de codes, commentaires et lignes blanches d'un projet. L'évaluation se lance via la commande de menu \menu{Extensions,Code statistics}.


\section{Profilage de Code}

Une interface graphique simple au Profileur GNU GProf.


\section{Importation de Projets}

Le plugin ProjectsImporter importe des projets et des espaces de travail d'autres IDE, notamment Dev-C++, MSVC6, MSVC7, et MSVC8 pour les utiliser comme projets \codeblocks. 


\section{Recherche de Code Source Disponible}

Cette extension permet de sélectionner un terme dans l'éditeur et de rechercher ce terme à l'aide du menu de contexte \menu{Rechercher dans Koders} dans la base de données du site \cite{url:koders}. La boîte de dialogue permet d'ajouter la possibilité de filtrer les langages de programmation ou le type de licence.

\hint{Koders et son successeur BlackDuck semblent avoir disparu ou changé de site Web ! Aussi ce plugin ne fonctionne plus. En attente de mise à jour ...}

Cette recherche dans la base vous aidera à trouver du code source originaire du monde entier en provenance d'autres projets universitaires, consortiums et d'organisations comme Apache, Mozilla, Novell Forge, SourceForge et bien d'autres, qui peuvent être réutilisés sans avoir à réinventer la roue à chaque fois. SVP, regardez bien la licence du code source dans chaque cas particulier.


\section{Extension Symbol Table}

Cette extension permet de rechercher des symboles dans des fichiers objets et dans des librairies. Les options et le chemin d'accès au programme nm en ligne de commande sont définis dans l'onglet des Options.

\screenshot{symtab_config}{Configuration de Symbol Table}

Cliquer sur \samp{Rechercher} démarre la recherche. Les résultats du programme NM sont alors affichés dans une fenêtre séparée nommée \samp{SymTabs Result}. Les noms des fichiers objets et des librairies contenant les symboles sont listés avec comme titre \samp{Sortie de NM}.

