\section{ToDo List}\label{sec:todo_list}

In complex software projects, where different users are involved, there is often the requirement of different tasks to be performed by different users. For this purpose, \codeblocks offers a Todo List. This list can be opened via \menu{View,To-Do list}, and contains the tasks to be performed, together with their priorities, types and the responsible users. The list can be filtered for tasks, users and/or source files. A sorting by columns can be achieved by clicking the caption of the corresponding column.

\screenshot{todo_list}{Displaying the ToDo List}

\hint{The To-Do list can be docked in the message console. Select the option \samp{Include the To-Do list in the message pane} via the menu \menu{Settings,Environment}.}

If the sources are opened in \codeblocks, a Todo can be added to the list via the context menu command \samp{Add To-Do item}. A comment will be added in the selected line of the source code.

\begin{lstlisting}
// TODO (user#1#): add new dialog for next release
\end{lstlisting}

When adding a To-Do, a dialogue box will appear where the following settings can be made (see \pxref{fig:add_todo}).

\figures[hbt!][width=.5\columnwidth]{add_todo}{Dialogue for adding a ToDo}

\begin{description}
\item[User] User name \var{user} in the operating system. Tasks for other users can also be created here. In doing so, the corresponding user name has to be created by Add new. The assignment of a Todo is then made via the selection of entries listed for the User.

\hint{Note that the Users have nothing to do with the Personalities used in \codeblocks.}
\item[Type] By default, type is set to Todo.
\item[Priority] The importance of tasks can be expressed by priorities (1 - 9) in \codeblocks.
\item[Position] This setting specifies whether the comment is to be included before, after or at the exact position of the cursor.
\item[Comment Style] A selection of formats for comments (e.g. doxygen).
\end{description}
