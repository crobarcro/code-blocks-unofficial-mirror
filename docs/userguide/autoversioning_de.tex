\section{AutoVersioning}\label{sec:autoversioning}

Ein Plugin zur Versionierung von Anwendungen, indem die Versions- und Buildnummer einer Anwendung jedesmal hochgezählt wird, wenn eine Änderung stattgefunden hat. Diese Information wird über einfach benutzbare Variablendeklarationen in der Datei \file{version.h} abgelegt. Des weiteren sind möglich: Übergaben im SVN Stil, ein Versionsschema Editor, ein Change Log Generator und ein Log Generator und vieles mehr $\ldots$

\subsection{Einleitung}

Die Idee dieses Plugins entstand bei Entwicklung von Software, die sich im frühen pre-alpha Status befand und eine Art von Versionsinformation benötigte. Beschäftigt durch die Erstellung von Code, blieb keine Zeit um die Versionsnummer zu pflegen, deshlab wurde ein Plugin entwickelt, dass diese Arbeit erledigt und nur minimaler Bedienereingriff erfordert.

\subsection{Features}

Hier finden Sie eine Liste von Features, die vom Plugin abgedeckt werden.

\begin{itemize}
\item Unterstützung für C und C++.
\item Generiert und auto inkrementiert Versionsvariablen.
\item Software status editor.
\item Integrierter Schemeneditor für die Konfiguration wie automatische Hochzählen der \samp{version values} geschehen soll.
\item Datum deklariert als Monat, Datum und Jahr.
\item Ubuntu style version.
\item Svn revision check.
\item Change log generator.
\item Funktioniert unter Windows und Linux.
\end{itemize}

\subsection{Handhabung}

Wählen Sie einfach das Menü \menu{Project,Autoversioning}. Das Pop Up Fenster wie auf \pxref{autoversion_configure} erscheint.

\figures[H][]{autoversion_configure}{Configure project for Autoversioning}

Wenn Sie den Dialog mit yes bestätigen, dann wird der Konfigurationsdialog von Autoversioning angezeigt.

Nachdem Sie Ihr Projekt für Autoversioning konfiguriert haben, werden die Einstellungen aus dem Eingabedialog im Projekt gespeichert und eine Datei \file{version.h} wird angelegt. Ab diesem Zeitpunkt wird bei jedem Aufruf des Menüs \menu{Project,Autoversioning} der Konfigurationsdialog aufgerufen, um die Einstellung für Projektversion vorzunehmen, es sei denn Sie speichern die Änderungen des Plugins in Projektdatei.

\subsection{Dialog notebook tabs}
\subsubsection{Version Values}

Hier können Sie einfach die zugehörigen Version Values eintragen oder Auswählen ob Auto Versioning diese für Sie hochzählt (siehe \pxref{fig:autoversion_editor}).

\begin{description}
\item[Major] Wird um eins hochgezählt wenn die Minor Version ihr Maximum erreicht
\item[Minor] Wird um eins hochgezählt wenn die Anzahl von Build die Schranke build times überschreitet. Der Wert wird auch Null zurückgesetzt, nachdem die maximale Anzahl bereits erreicht wurde.
\item[Build Number] Gleichbedeutend mit Release und wird jedesmal wenn die Revison Number steigt um eins hochgezählt.
\item[Revision] Zählt die Revision zufallsartig hoch, wenn das Projekt geändert oder kompiliert wurde.
\end{description}

\figures{autoversion_editor}{Set Version Values}

\subsubsection{Status}

Einige Felder sind auf vordefiniert Werte voreingestellt (siehe \pxref{fig:autoversion_status}).

\begin{description}
\item[Software Status] Ein typisches Beispiel wäre v1.0 Alpha
\item[Abbreviation] Gleichbedeutend mit Software Status, aber in der Form: v1.0a
\end{description}

\figures{autoversion_status}{Setzen Status von Autoversioning}

\subsubsection{Scheme}

Hier stellen Sie ein, wie das Plugin die version values hochzählt (siehe \pxref{fig:autoversion_scheme}).

\figures{autoversion_scheme}{Scheme of autoversioning}

\begin{description}
\item[Minor maximum] Die obere Schranke für den Wert Minor. Wenn diese erreicht ist, wird Major hochgezählt und beim nächsten Kompilevorgang des Projektes wird Minor auf Null zurückgesetzt.
\item[Build Number maximum] Wenn der Wert erreicht wurde, wird beim nächsten Kompilevorgang der Wert auf Null zurückgesetzt. Die Einstellung 0 setzt das Maximum auf unendlich
\item[Revision maximum] Gleichbedeutend mit Maximum für Build Number maximum. Die Einstellung 0 setzt das Maximum auf unendlich
\item[Revision random maximum] Die Revisions Nummer wird durch Zufallszahlen hochgezählt. Eine Einstellung mit 1, wird die Revision um eins erhöhen.
\item[Build times before incrementing Minor] Nach Änderungen im Code und Kompilierung wird die Build History inkrementiert und wenn dieser Wert erreicht wird, dann wir der Minor Wert auch inkrementiert.
\end{description}

\subsubsection{Einstellungen}

Hier können Sie einige Einstellungen für Auto Versioning vornehmen (siehe \pxref{fig:autoversion_settings}).

\figures{autoversion_settings}{Settings von Autoversioning}

\begin{description}
\item[Autoincrement Major and Minor] Lassen Sie das Plugin nach diesem Schema den Wert inkrementieren. Wenn es nicht ausgewählt wurde, dann wird nur die Build Number und die Revision hochgezählt.
\item[Create date declarations] Erzeugt in der Datei \file{version.h} Einträge für Datum und Ubuntu style version.
\item[Do Auto Increment] Weist das Plugin an bei jeder Änderung noch vor dem Kompilevorgang zu inkrementieren.
\item[Header language] Einstellung der Sprache für Ausgabe in \file{version.h}
\item[Ask to increment] Wenn Do Auto Increment aktiv ist, wird vor dem Kompilevorgang bei Änderungen nachgefragt, ob hochgezählt werden soll.
\item[svn enabled] Sucht nach der SVN Revision und Datum im aktuellen Verzeichnis und erzeugt die zugehörigen Einträge in \file{version.h}
\end{description}

\subsubsection{Changes Log}

Durch diese Einstellung wird die Eingabe für jegliche Änderung an einem Projekt in die Datei \file{ChangesLog.txt} generiert (siehe \pxref{fig:autoversion_changelog}).

\figures{autoversion_changelog}{Changelog von Autoversioning}

\begin{description}
\item[Show changes editor when incrementing version] Ruft den Changes log editor auf, wenn die Version inkrementiert wird.
\item[Title Format] Format für Title mit einer Liste von vordefinierten Werten.
\end{description}

\subsection{Einbinden in den Quellen}

Für die Verwendung der Variablen, die durch das Plugin erzeugt wurden, müssen Sie die Datei \codeline{\#include <version.h>} in den Quellen einfügen. Ein Beispiel für eine Quelle könnte wie folgt aussehen:

\begin{code}
#include <iostream>
#include "version.h"

void main(){
    std::cout<<AutoVersion::Major<<endl;
}
\end{code}

\subsubsection{Ausgabe von version.h}

Die erzeugte Headerdatei könnte beispielsweise im C++ Mode wie folgt aussehen:

\begin{code}
#ifndef VERSION_H
#define VERSION_H

namespace AutoVersion{

	//Date Version Types
	static const char DATE[] = "15";
	static const char MONTH[] = "09";
	static const char YEAR[] = "2007";
	static const double UBUNTU_VERSION_STYLE = 7.09;

	//Software Status
	static const char STATUS[] = "Pre-alpha";
	static const char STATUS_SHORT[] = "pa";

	//Standard Version Type
	static const long MAJOR = 0;
	static const long MINOR = 10;
	static const long BUILD = 1086;
	static const long REVISION = 6349;

	//Miscellaneous Version Types
	static const long BUILDS_COUNT = 1984;
	#define RC_FILEVERSION 0,10,1086,6349
	#define RC_FILEVERSION_STRING "0, 10, 1086, 6349\0"
	static const char FULLVERSION_STRING[] = "0.10.1086.6349";

}
#endif //VERSION_h
\end{code}

Bei der Einstellung der Sprache C ergibt sich folgende Ausgabe ohne Namespaces:

\begin{code}
#ifndef VERSION_H
#define VERSION_H

	//Date Version Types
	static const char DATE[] = "15";
	static const char MONTH[] = "09";
	static const char YEAR[] = "2007";
	static const double UBUNTU_VERSION_STYLE = 7.09;

	//Software Status
	static const char STATUS[] = "Pre-alpha";
	static const char STATUS_SHORT[] = "pa";

	//Standard Version Type
	static const long MAJOR = 0;
	static const long MINOR = 10;
	static const long BUILD = 1086;
	static const long REVISION = 6349;

	//Miscellaneous Version Types
	static const long BUILDS_COUNT = 1984;
	#define RC_FILEVERSION 0,10,1086,6349
	#define RC_FILEVERSION_STRING "0, 10, 1086, 6349\0"
	static const char FULLVERSION_STRING[] = "0.10.1086.6349";

#endif //VERSION_h
\end{code}

\subsection{Change log generator}

Dieser Dialog ist über das Menü \menu{Project,Changes Log} erreichbar. Diese Dialog erscheint auch wenn die Einstellung \menu{Show changes editor} für das Inkrementierung der Version (Changes Log) besteht. Im Dialog werden die Liste von Änderungen nach Modifikation der Quellen eines Projektes eingegeben (siehe \pxref{fig:autoversion_changes}).

\figures[H][]{autoversion_changes}{Changes for a project}

\subsubsection{Buttons Summary}

\begin{description}
\item[Add] Fügt eine Zeile in der Liste hinzu
\item[Edit] Editieren einer ausgwählte Zelle
\item[Delete] Löscht die ausgewählte Zeile aus der Liste
\item[Save] Speichert die aktuellen Daten temporär in der Datei (\file{changes.tmp}). Diese Information wird später für die Ausgabe in Changes Log verwendet
\item[Write] Speichert die Eingabe in der Changes Log Datei
\item[Cancel] Beendet den Dialog
\end{description}

Hier ein Beispiel für eine Datei \file{ChangesLog.txt}, die durch Auto Versioning erzeugt wurde.

\begin{code}
03 September 2007
   released version 0.7.34 of AutoVersioning-Linux

     Change log:
        -Fixed: pointer declaration
        -Bug: blah blah

02 September 2007
   released version 0.7.32 of AutoVersioning-Linux

     Change log:
        -Documented some areas of the code
        -Reorganized the code for readability

01 September 2007
   released version 0.7.30 of AutoVersioning-Linux

     Change log:
        -Edited the change log window
        -If the change log windows is leave blank no changes.txt is modified
\end{code}
