\section{Browse Tracker}\label{sec:browsetracker}

Browse Tracker is a plug-in that helps navigating between recently opened files in \codeblocks. The list of recent files is saved in a history. With the menu \menu{View,Browse Tracker,Clear All} the history is cleared.

With the window \samp{Browsed Tabs} you can navigate between the items of the recently opened files using the menu entry \menu{View,Browse Tracker,Backward Ed/Forward Ed} or the shortcut Alt-Left/Alt-Right. The Browse Tracker menu is also accessible as context menu. The markers are saved in the layout file \file{\var{projectName}.bmarks}

A common procedure when developing software is to struggle with a set of functions which are implemented in different files. The BrowseTracks plug-in will help you solve this problem by showing you the order in which the files were selected. You can then comfortably navigate the function calls.

The plug-in allows even browse markers within each file in the \codeblocks editor. The cursor position is memorized for every file. You can set this markers using the menu item \menu{View, Browse Tracker, Set BrowseMarks} or with selecting a line with the left mouse button. A marker with $\ldots$ is shown in the left margin. With the menu \menu{View,Browse Tracker,Prev Mark/Next Mark} or the shortcut Alt-up/Alt-down you can navigate through the markers within a file. If you want to navigate in a file between markers sorted by line numbers then just select the menu \menu{View,Browse Tracker,Sort BrowseMark}.

With the \menu{Clear BrowseMark} the marker in a selected line is removed. If a marker is set for a line, holding left-mouse button down for 1/4 second while pressing the Ctrl key will delete the marker for this line. Via the menu \menu{Clear All BrowseMarks} or with a Ctrl-left click on any unmarked line will reset the markers within a file.

The settings of the plug-in can be configure via the menu \menu{Settings,Editor,Browse Tracker}.

\begin{description}
\item[Mark Style] Browse Marks are displayed per default as $\ldots$ within the margin. With the setting \menu{Book\_Marks} they will be displayed like Bookmarks as blue arrow in the margin. With hide the display of Browse Marks is suppressed.
\item[Toggle Browse Mark key] Markers can be set or removed either by a click with the left mouse button or with a click while holding the crtl key.
\item[Toggle Delay] The duration of holding the left mouse button to enter the Browse Marker mode.
\item[Clear All BrowseMarks] while holding Ctrl key either by a simple or a double click with the left mouse button.
\end{description}

The configuration of the plug-in is stored in your application data directory in the file \file{default.conf}. If you use the personality feature of \codeblocks the configuration is read from the file \file{\var{personality}.conf}.






