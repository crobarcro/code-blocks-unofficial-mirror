\section{Code statistics}

\screenshot{code_stats}{Konfiguration für Code Statistik}

Anhand der Angaben in der Konfigurationsmaske ermittelt dieses einfache Plugin die Anteile von Code, Kommentaren und Leerzeilen für ein Projekt. Die Auswertung wird über das Menü \menu{Plugins,Code statistics} durchgeführt.

\section{Suche nach verfügbaren Quellencode}

Dieses Plugin ermöglicht es, einen Begriff im Editor zu markieren und über das Kontextmenü \menu{Search at Koders} in der Datenbank von \cite{url:koders} zu suchen. Dabei bietet der Eingabedialog zusätzlich die Möglichkeit, die Suche nach Programmiersprachen und Lizenzen zu filtern.

Durch diese Datenbanksuche finden Sie schnell Quellcode der aus anderen weltweiten Projekten von Universitäten, Consortiums und Organisationen wie Apache, Mozilla, Novell Forge, SourceForge und vielen mehr stammt und wiederverwendet werden kann, ohne dass jedes Mal das Rad neu erfunden werden muss. Bitte beachten Sie die jeweilige Lizenz des Quellcodes.

\section{Code profiler}

Eine einfache grafische Schnittstelle für das Profiler Programm GNU GProf.

\section{Symbol Table Plugin}

Diese Plugin ermöglicht die Suche von Symbolen in Objekten und Bibliotheken. Dabei werden die Optionen und der Pfad für das Kommandozeilen Programm nm über den Reiter Options konfiguriert.

\screenshot{symtab_config}{Konfiguration von Symbol Table}

Mit der Schaltfläche \samp{Search} wird die Suche gestartet und die Ergebnisse des Programms NM werden in einem eigenen Fenster SymTabs Result angezeigt. Der Name des Objekts bzw. Bibliothek, die das Symbol enthalten ist unter dem Titel NM's Output gelistet.
