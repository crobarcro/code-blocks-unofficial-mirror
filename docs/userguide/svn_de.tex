\section{SVN Unterstützung}\label{sec:svn}

Eine Unterstützung für die SVN Versionskontrolle bietet das \codeblocks Plugin TortoiseSVN. Im Menü \menu{TortoiseSVN,Plugin settings} lässt sich im Reiter \menu{Integration} einstellen, wo die benutzerdefinierbaren SVN-Befehlen zur Verfügung stehen sollen.

\begin{description}
\item[Menu integration] Fügt einen Eintrag TortoiseSVN mit Einstellmöglichkeiten in die Menüleiste ein.
\item[Project manger] Aktiviert die TortoiseSVN Befehle im Kontextmenü des Project Managers.
\item[Editor] Aktiviert die TortoiseSVN Befehle im Kontextmenü des Editors.
\end{description}

In den Plugin Settings lässt sich zusätzlich im Integration Dialog \menu{Edit main menu} beziehungsweise \menu{Edit popup menu} konfigurieren welche SVN Kommandos im Menü bzw. Kontextmenü ausgeführt werden können.

\hint{Der File Explorer in \codeblocks bietet für die Anzeige des SVN Status unterschiedliche Farben als Icon Overlays. Auch hier ist das Kontextmenü für TortoiseSVN zugänglich.}
