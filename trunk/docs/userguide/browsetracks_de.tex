\section{BrowseTracks}\label{sec:browsetracks}

BrowseTracker ist ein Plugin um zwischen kürzlich geöffneten Dateien in \codeblocks zu navigieren. Dabei wird die Liste der kürzlich geöffneten Dateien in einer History gespeichert. Derzeit wird eine Liste mit maximal 20 Einträge unterstützt. Im Menü \menu{View,Browse Tracker,Clear} können Sie die History löschen.

Das Fenster \samp{Browsed Tabs} zum Navigieren in dieser Listen erhalten Sie über das Menü \menu{View,Browse Tacker} mit dem Eintrag \samp{forward}/\samp{backward} oder über das Tastenkürzel Alt-Left/Alt-Right.

Eine häufige Arbeitsweise bei der Entwicklung von Software ist, dass man sich durch ein Satz von Funktion hangelt, die in unterschiedlichen Dateien implementiert sind. Durch das Plugin BrowseTracks können Sie somit komfortabel zwischen den Aufrufen in unterschiedlichen Dateien navigieren.

Die Konfiguration für das Plugin wird in den Anwendungsdaten in der Datei \file{default.conf} gepseichert. Bei der Verwenundung einer Personality wird die Konfiguration aus der Datei \file{\var{personality}.conf} gelesen.
